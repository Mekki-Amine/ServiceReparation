\documentclass[aspectratio=169,12pt]{beamer}
\usepackage[utf8]{inputenc}
\usepackage[french]{babel}
\usepackage{graphicx}
\usepackage{tikz}
\usepackage{listings}
\usepackage{xcolor}
\usetikzlibrary{shapes,arrows,positioning,fit,calc,decorations.pathmorphing}

% Thème
\usetheme{Madrid}
\usecolortheme{default}

% Couleurs personnalisées
\definecolor{fixeryellow}{RGB}{255,193,7}
\definecolor{fixergray}{RGB}{51,51,51}

% Informations de la présentation
\title[Fixer - PFE]{Fixer\\Plateforme de Services de Réparation d'Électroménagers}
\subtitle{Projet de Fin d'Études}
\author{Amine Mekki}
\institute{
  \textbf{Université SESAME}\\
  Encadrant Académique : Sameh BENNOUR\\
  Encadrant Entreprise : Kais EUCHI (C.C.A)
}
\date{Année Universitaire 2024-2025}

\begin{document}

% Page de titre
\frame{\titlepage}

% Table des matières
\begin{frame}{Plan de la présentation}
\tableofcontents
\end{frame}

% ============================================
% INTRODUCTION
% ============================================
\section{Introduction}

\begin{frame}{Introduction Générale}
\begin{itemize}
\item \textbf{Contexte :} Digitalisation du secteur des services de réparation
\item \textbf{Problématique :} Difficultés de mise en relation entre clients et réparateurs
\item \textbf{Solution :} Plateforme web moderne de gestion de services de réparation
\item \textbf{Objectif :} Faciliter l'accès aux services de réparation d'électroménagers
\end{itemize}

\vspace{0.5cm}
\begin{block}{Valeur ajoutée}
\begin{itemize}
\item Interface intuitive et moderne
\item Sécurité renforcée (JWT)
\item Architecture modulaire et évolutive
\item Expérience utilisateur optimale
\end{itemize}
\end{block}
\end{frame}

% ============================================
% CHAPITRE 1: CONTEXTE
% ============================================
\section{Contexte et Cadre du Projet}

\begin{frame}{Présentation de l'organisme d'accueil}
\begin{columns}
\column{0.5\textwidth}
\begin{block}{CCA}
\begin{itemize}
\item Startup tunisienne fondée en 2010
\item Spécialisée en développement web
\item Secteur : Communication et marketing digital
\item Siège : 97 Avenue Habib Bourguiba, Ariana
\item Effectif : 10 employés
\item Directeur/CEO : Youssef Berhouma
\end{itemize}
\end{block}

\column{0.5\textwidth}
\begin{block}{Mission}
Fournir des solutions digitales innovantes et des services de communication sur mesure pour aider les entreprises à renforcer leur présence en ligne et à développer leur notoriété.
\end{block}
\end{columns}
\end{frame}

\begin{frame}{Cadre du projet}
\begin{block}{Objectifs fonctionnels}
\begin{itemize}
\item Développer une plateforme complète de gestion de services
\item Permettre la publication et consultation de demandes
\item Faciliter la communication entre utilisateurs
\item Gérer les transactions et paiements
\end{itemize}
\end{block}

\begin{block}{Objectifs techniques}
\begin{itemize}
\item Architecture modulaire et évolutive
\item Sécurité renforcée (JWT)
\item Interface responsive et moderne
\item Performance et disponibilité élevées
\end{itemize}
\end{block}
\end{frame}

\begin{frame}{Méthodologie de travail}
\begin{columns}
\column{0.5\textwidth}
\begin{block}{Approche Agile}
\begin{itemize}
\item Développement par sprints
\item Itérations courtes (2-4 semaines)
\item Adaptation continue
\item Livraison incrémentale
\end{itemize}
\end{block}

\column{0.5\textwidth}
\begin{block}{Modélisation UML}
\begin{itemize}
\item Diagramme de cas d'utilisation
\item Diagramme de classes
\item Diagramme de séquence
\item Diagramme d'architecture
\end{itemize}
\end{block}
\end{columns}

\vspace{0.5cm}
\begin{center}
\textbf{4 Sprints} répartis sur la durée du stage
\end{center}
\end{frame}

% ============================================
% CHAPITRE 2: ANALYSE
% ============================================
\section{Analyse et Spécification}

\begin{frame}{Besoins fonctionnels}
\begin{columns}
\column{0.5\textwidth}
\begin{block}{Client}
\begin{itemize}
\item Authentification sécurisée
\item Publication de demandes
\item Consultation du catalogue
\item Messagerie
\item Panier d'achat
\item Notifications
\end{itemize}
\end{block}

\column{0.5\textwidth}
\begin{block}{Administrateur}
\begin{itemize}
\item Validation des publications
\item Gestion des utilisateurs
\item Gestion des messages
\item Statistiques et rapports
\end{itemize}
\end{block}
\end{columns}
\end{frame}

\begin{frame}{Diagramme de cas d'utilisation global}
\vspace{-0.3cm}
\centering
\resizebox{0.9\textwidth}{!}{%
\begin{tikzpicture}[
    actor/.style={rectangle, draw=black, fill=blue!30, text width=2cm, text centered, rounded corners, minimum height=1cm, font=\normalsize\bfseries},
    usecase/.style={ellipse, draw=black, fill=yellow!30, text width=2.2cm, text centered, minimum height=0.7cm, font=\footnotesize},
    system/.style={rectangle, draw=black, dashed, fill=gray!10, text width=12cm, text height=8cm, inner sep=0.5cm}
]

% Système
\node[system] (system) at (0,0) {};
\node[font=\large\bfseries] at (0,3.8) {Système Fixer};

% Acteurs
\node[actor] (client) at (-6,2.5) {Client};
\node[actor] (admin) at (-6,-2.5) {Administrateur};

% Cas d'utilisation - Authentification
\node[usecase] (inscrire) at (-2,5.5) {S'inscrire};
\node[usecase] (authentifier) at (0.5,5.5) {S'authentifier};
\node[usecase] (profil) at (3,5.5) {Gérer profil};

% Cas d'utilisation - Publications
\node[usecase] (publier) at (-4,2.5) {Publier demande};
\node[usecase] (consulter) at (-2,2.5) {Consulter catalogue};
\node[usecase] (rechercher) at (0.5,2.5) {Rechercher};
\node[usecase] (modifier) at (3,2.5) {Modifier publication};

% Cas d'utilisation - Communication
\node[usecase] (message) at (-2,-0.5) {Envoyer message};
\node[usecase] (notification) at (0.5,-0.5) {Recevoir notifications};

% Cas d'utilisation - Panier
\node[usecase] (panier) at (-4,-0.5) {Gérer panier};
\node[usecase] (paiement) at (-6.5,-0.5) {Effectuer paiement};

% Cas d'utilisation - Admin
\node[usecase] (valider) at (-2,-3.5) {Valider publication};
\node[usecase] (gerer) at (0.5,-3.5) {Gérer utilisateurs};
\node[usecase] (stats) at (3,-3.5) {Consulter statistiques};

% Relations Client
\draw[->, line width=1pt] (client) -- (inscrire);
\draw[->, line width=1pt] (client) -- (authentifier);
\draw[->, line width=1pt] (client) -- (profil);
\draw[->, line width=1pt] (client) -- (publier);
\draw[->, line width=1pt] (client) -- (consulter);
\draw[->, line width=1pt] (client) -- (rechercher);
\draw[->, line width=1pt] (client) -- (modifier);
\draw[->, line width=1pt] (client) -- (message);
\draw[->, line width=1pt] (client) -- (notification);
\draw[->, line width=1pt] (client) -- (panier);
\draw[->, line width=1pt] (client) -- (paiement);

% Relations Admin
\draw[->, line width=1pt] (admin) -- (valider);
\draw[->, line width=1pt] (admin) -- (gerer);
\draw[->, line width=1pt] (admin) -- (stats);
\draw[->, line width=1pt] (admin) -- (authentifier);
\draw[->, line width=1pt] (admin) -- (consulter);

\end{tikzpicture}%
}
\end{frame}

\begin{frame}{Architecture générale}
\begin{block}{Architecture en 3 couches}
\begin{enumerate}
\item \textbf{Couche Présentation} : React.js 18
   \begin{itemize}
   \item Pages, Composants, Context API
   \item Tailwind CSS, React Router
   \end{itemize}
\item \textbf{Couche Logique Métier} : Spring Boot 3.4.3
   \begin{itemize}
   \item Controllers, Services, Repositories
   \item Spring Security, JWT
   \end{itemize}
\item \textbf{Couche Données} : MySQL 8.0+
   \begin{itemize}
   \item Tables : utilisateur, publication, message, etc.
   \end{itemize}
\end{enumerate}
\end{block}
\end{frame}

\begin{frame}{Architecture Logicielle}
\vspace{-0.3cm}
\centering
\resizebox{0.92\textwidth}{!}{%
\begin{tikzpicture}[
    layer/.style={rectangle, rounded corners=10pt, minimum width=12cm, minimum height=3cm, 
                  text centered, font=\normalsize, draw=black, line width=2pt},
    component/.style={rectangle, rounded corners=5pt, minimum width=2.5cm, minimum height=1cm, 
                      text centered, font=\tiny, draw=black, fill=white},
    arrow/.style={->, line width=2pt, color=blue!70}
]

% Définition des couleurs
\definecolor{frontendcolor}{RGB}{227,242,253}
\definecolor{backendcolor}{RGB}{232,245,233}
\definecolor{dbcolor}{RGB}{255,243,224}

% COUCHE PRÉSENTATION
\node[layer, fill=frontendcolor] (frontend) at (0,6) {};
\node[font=\large\bfseries] at (0,7) {COUCHE PRÉSENTATION};
\node[font=\normalsize] at (0,6.5) {Frontend - React.js 18};

% Composants Frontend
\node[component] (pages) at (-3.5,6) {Pages\\React Router};
\node[component] (components) at (-1,6) {Composants};
\node[component] (context) at (1,6) {Context API};
\node[component] (api) at (3.5,6) {api.js\\Axios};

% Flèche Frontend → Backend
\draw[arrow] (0,5) -- (0,3.5);
\node[font=\small\bfseries, color=blue!70] at (0.5,4.2) {HTTP/REST API};

% COUCHE LOGIQUE MÉTIER
\node[layer, fill=backendcolor] (backend) at (0,1.5) {};
\node[font=\large\bfseries] at (0,2.5) {COUCHE LOGIQUE MÉTIER};
\node[font=\normalsize] at (0,2) {Backend - Spring Boot 3.4.3};

% Composants Backend
\node[component] (controllers) at (-3.5,1.5) {Controllers\\REST API};
\node[component] (services) at (-1,1.5) {Services\\Métier};
\node[component] (repos) at (1,1.5) {Repositories\\JPA};
\node[component] (security) at (3.5,1.5) {Security\\JWT};

% Flèches entre composants Backend
\draw[->, line width=1pt, color=gray] (controllers) -- (services);
\draw[->, line width=1pt, color=gray] (services) -- (repos);

% Flèche Backend → Database
\draw[arrow] (0,0.5) -- (0,-1);
\node[font=\small\bfseries, color=blue!70] at (0.5,-0.2) {JPA/Hibernate};

% COUCHE DONNÉES
\node[layer, fill=dbcolor] (database) at (0,-2.5) {};
\node[font=\large\bfseries] at (0,-1.5) {COUCHE DONNÉES};
\node[font=\normalsize] at (0,-2) {Base de données - MySQL 8.0+};

% Tables
\node[component] (tables) at (0,-2.5) {Tables Principales};
\node[font=\tiny, color=gray] at (0,-3.5) {
    utilisateur, publication, message,\\
    notification, cart, cart\_item
};

\end{tikzpicture}%
}
\end{frame}

\begin{frame}[fragile]{Diagramme de classe global}
\vspace{-0.3cm}
\centering
\resizebox{0.92\textwidth}{!}{%
\begin{tikzpicture}[
    class/.style={rectangle, draw=black, fill=white, text width=3cm, text centered, minimum height=2.2cm, font=\footnotesize, align=left, inner sep=4pt, line width=1.5pt}
]

% Classe Utilisateur
\node[class] (user) at (-4.5,3.5) {
    \textbf{Utilisateur}\\
    \textcolor{gray}{\rule{2.8cm}{0.5pt}}\\
    {\small\color{gray}- id: Long}\\
    {\small\color{gray}- username: String}\\
    {\small\color{gray}- email: String}\\
    {\small\color{gray}- role: Role}\\
    {\small\color{gray}- profilePhoto: String}\\
    \textcolor{gray}{\rule{2.8cm}{0.5pt}}\\
    {\small\color{blue}+ authenticate()}\\
    {\small\color{blue}+ updateProfile()}
};

% Classe Publication
\node[class] (pub) at (0,3.5) {
    \textbf{Publication}\\
    \textcolor{gray}{\rule{2.8cm}{0.5pt}}\\
    {\small\color{gray}- id: Long}\\
    {\small\color{gray}- title: String}\\
    {\small\color{gray}- description: String}\\
    {\small\color{gray}- type: String}\\
    {\small\color{gray}- price: Double}\\
    {\small\color{gray}- verified: Boolean}\\
    \textcolor{gray}{\rule{2.8cm}{0.5pt}}\\
    {\small\color{blue}+ create()}\\
    {\small\color{blue}+ update()}
};

% Classe Message
\node[class] (message) at (4.5,3.5) {
    \textbf{Message}\\
    \textcolor{gray}{\rule{2.8cm}{0.5pt}}\\
    {\small\color{gray}- id: Long}\\
    {\small\color{gray}- content: String}\\
    {\small\color{gray}- fileUrl: String}\\
    {\small\color{gray}- latitude: Double}\\
    {\small\color{gray}- longitude: Double}\\
    \textcolor{gray}{\rule{2.8cm}{0.5pt}}\\
    {\small\color{blue}+ send()}\\
    {\small\color{blue}+ delete()}
};

% Classe Notification
\node[class] (notif) at (-4.5,0.5) {
    \textbf{Notification}\\
    \textcolor{gray}{\rule{2.8cm}{0.5pt}}\\
    {\small\color{gray}- id: Long}\\
    {\small\color{gray}- message: String}\\
    {\small\color{gray}- isRead: Boolean}\\
    {\small\color{gray}- createdAt: Date}\\
    \textcolor{gray}{\rule{2.8cm}{0.5pt}}\\
    {\small\color{blue}+ markAsRead()}
};

% Classe Cart
\node[class] (cart) at (0,0.5) {
    \textbf{Cart}\\
    \textcolor{gray}{\rule{2.8cm}{0.5pt}}\\
    {\small\color{gray}- id: Long}\\
    {\small\color{gray}- total: Double}\\
    \textcolor{gray}{\rule{2.8cm}{0.5pt}}\\
    {\small\color{blue}+ addItem()}\\
    {\small\color{blue}+ removeItem()}\\
    {\small\color{blue}+ calculateTotal()}
};

% Classe CartItem
\node[class] (cartitem) at (4.5,0.5) {
    \textbf{CartItem}\\
    \textcolor{gray}{\rule{2.8cm}{0.5pt}}\\
    {\small\color{gray}- id: Long}\\
    {\small\color{gray}- quantity: Integer}\\
    \textcolor{gray}{\rule{2.8cm}{0.5pt}}\\
    {\small\color{blue}+ updateQuantity()}
};

% Relations
\draw[->, line width=2pt, color=red!70] (user) -- node[above, font=\small\bfseries] {1..*} (pub);
\draw[->, line width=2pt, color=red!70] (user) -- node[above, font=\small\bfseries] {1..*} (message);
\draw[->, line width=2pt, color=red!70] (user) -- node[left, font=\small\bfseries] {1} node[right, font=\small\bfseries] {1} (cart);
\draw[->, line width=2pt, color=red!70] (user) -- node[left, font=\small\bfseries] {1..*} (notif);
\draw[->, line width=2pt, color=red!70] (cart) -- node[above, font=\small\bfseries] {1..*} (cartitem);
\draw[->, line width=2pt, color=red!70] (pub) -- node[left, font=\small\bfseries] {1..*} (cartitem);

\end{tikzpicture}%
}
\end{frame}

\begin{frame}{Stack technologique}
\begin{center}
\begin{tabularx}{\textwidth}{|X|X|}
\hline
\textbf{Frontend} & React.js 18, React Router v6, Tailwind CSS, Axios, Vite \\
\hline
\textbf{Backend} & Spring Boot 3.4.3, Spring Security, Spring Data JPA, Java 21 \\
\hline
\textbf{Base de données} & MySQL 8.0+, InnoDB Engine \\
\hline
\textbf{Sécurité} & JWT, BCrypt, Spring Security \\
\hline
\textbf{Outils} & IntelliJ IDEA, VS Code, Postman, Git \\
\hline
\end{tabularx}
\end{center}
\end{frame}

% ============================================
% SPRINT 1
% ============================================
\section{Sprint 1}

\begin{frame}{Sprint 1 - Authentification}
\begin{block}{Objectifs}
\begin{itemize}
\item Système d'authentification sécurisé
\item Gestion des rôles (CLIENT, ADMIN)
\item Interface d'inscription et de connexion
\end{itemize}
\end{block}

\begin{block}{Fonctionnalités développées}
\begin{itemize}
\item Inscription avec validation
\item Connexion avec JWT
\item Gestion du profil utilisateur
\item Protection des routes selon les rôles
\end{itemize}
\end{block}
\end{frame}

\begin{frame}{Sprint 1 - Interfaces}
\begin{columns}
\column{0.5\textwidth}
\begin{block}{Inscription}
\begin{itemize}
\item Formulaire complet
\item Validation des champs
\item Upload photo de profil
\item Gestion des erreurs
\end{itemize}
\end{block}

\column{0.5\textwidth}
\begin{block}{Authentification}
\begin{itemize}
\item Connexion sécurisée
\item Génération de token JWT
\item Redirection selon le rôle
\item Gestion des sessions
\end{itemize}
\end{block}
\end{columns}
\end{frame}

% ============================================
% SPRINT 2
% ============================================
\section{Sprint 2}

\begin{frame}{Sprint 2 - Publications et Catalogue}
\begin{block}{Objectifs}
\begin{itemize}
\item Système de publication de demandes
\item Catalogue de publications vérifiées
\item Gestion des publications par l'admin
\end{itemize}
\end{block}

\begin{block}{Fonctionnalités développées}
\begin{itemize}
\item Création de publications (titre, description, type, prix, fichier)
\item Upload d'images et documents
\item Filtrage et recherche dans le catalogue
\item Validation/Refus des publications par l'admin
\end{itemize}
\end{block}
\end{frame}

\begin{frame}{Sprint 2 - Interfaces}
\begin{block}{Catalogue}
\begin{itemize}
\item Affichage des publications vérifiées
\item Filtres avancés (type, prix, statut)
\item Recherche par mots-clés
\item Tri des résultats
\item Design responsive
\end{itemize}
\end{block}

\begin{block}{Gestion Admin}
\begin{itemize}
\item Interface de validation des publications
\item Filtres et recherche avancée
\item Actions : Valider, Refuser, Modifier, Supprimer
\end{itemize}
\end{block}
\end{frame}

% ============================================
% SPRINT 3
% ============================================
\section{Sprint 3}

\begin{frame}{Sprint 3 - Panier et Messagerie}
\begin{block}{Objectifs}
\begin{itemize}
\item Système de panier d'achat
\item Système de messagerie entre utilisateurs
\item Gestion des utilisateurs par l'admin
\end{itemize}
\end{block}

\begin{block}{Fonctionnalités développées}
\begin{itemize}
\item Ajout/Suppression d'articles au panier
\item Calcul du total
\item Envoi et réception de messages
\item Partage de localisation
\item Notifications en temps réel
\end{itemize}
\end{block}
\end{frame}

\begin{frame}{Sprint 3 - Interfaces}
\begin{columns}
\column{0.5\textwidth}
\begin{block}{Panier}
\begin{itemize}
\item Affichage des articles
\item Modification des quantités
\item Calcul automatique
\item Simulation de paiement
\end{itemize}
\end{block}

\column{0.5\textwidth}
\begin{block}{Messagerie}
\begin{itemize}
\item Interface de chat
\item Historique des conversations
\item Partage de localisation
\item Notifications nouvelles messages
\end{itemize}
\end{block}
\end{columns}
\end{frame}

% ============================================
% SPRINT 4
% ============================================
\section{Sprint 4}

\begin{frame}{Sprint 4 - Notifications et Chatbot}
\begin{block}{Objectifs}
\begin{itemize}
\item Système de notifications en temps réel
\item Chatbot intelligent pour la recherche
\item Finalisation de l'application
\end{itemize}
\end{block}

\begin{block}{Fonctionnalités développées}
\begin{itemize}
\item Notifications automatiques (validation, messages, etc.)
\item Chatbot avec recherche intelligente
\item Scoring de pertinence des résultats
\item Gestion des synonymes
\end{itemize}
\end{block}
\end{frame}

\begin{frame}{Sprint 4 - Interfaces}
\begin{columns}
\column{0.5\textwidth}
\begin{block}{Notifications}
\begin{itemize}
\item Affichage en temps réel
\item Badge de compteur
\item Marquage comme lu
\item Historique complet
\end{itemize}
\end{block}

\column{0.5\textwidth}
\begin{block}{Chatbot}
\begin{itemize}
\item Interface de chat intégrée
\item Recherche intelligente
\item Suggestions pertinentes
\item Gestion des synonymes
\end{itemize}
\end{block}
\end{columns}
\end{frame}

% ============================================
% RÉSULTATS
% ============================================
\section{Résultats et Réalisations}

\begin{frame}{Fonctionnalités réalisées}
\begin{block}{Côté Client}
\begin{itemize}
\item ✓ Authentification sécurisée (JWT)
\item ✓ Publication de demandes avec fichiers
\item ✓ Catalogue avec filtres avancés
\item ✓ Panier d'achat fonctionnel
\item ✓ Système de messagerie complet
\item ✓ Notifications en temps réel
\item ✓ Chatbot intelligent
\end{itemize}
\end{block}

\begin{block}{Côté Administrateur}
\begin{itemize}
\item ✓ Dashboard complet
\item ✓ Gestion des publications
\item ✓ Gestion des utilisateurs
\item ✓ Gestion des messages
\end{itemize}
\end{block}
\end{frame}


\begin{frame}{Statistiques du projet}
\begin{center}
\begin{tabularx}{\textwidth}{|X|X|}
\hline
\textbf{Composants React} & \textbf{15+ composants réutilisables} \\
\hline
\textbf{Endpoints API REST} & \textbf{25+ endpoints} \\
\hline
\textbf{Entités de base de données} & \textbf{8 entités principales} \\
\hline
\textbf{Pages développées} & \textbf{12+ pages} \\
\hline
\textbf{Sprints réalisés} & \textbf{4 sprints} \\
\hline
\textbf{Durée du projet} & \textbf{12-16 semaines} \\
\hline
\end{tabularx}
\end{center}
\end{frame}

% ============================================
% CONCLUSION
% ============================================
\section{Conclusion}

\begin{frame}{Conclusion}
\begin{block}{Réalisations}
\begin{itemize}
\item Plateforme web complète et fonctionnelle
\item Architecture modulaire et évolutive
\item Sécurité renforcée avec JWT
\item Interface moderne et responsive
\item Toutes les fonctionnalités prévues implémentées
\end{itemize}
\end{block}

\begin{block}{Compétences acquises}
\begin{itemize}
\item Développement full-stack (React, Spring Boot)
\item Modélisation UML
\item Gestion de projet agile
\item Sécurité des applications web
\item Travail en équipe
\end{itemize}
\end{block}
\end{frame}

\begin{frame}{Perspectives d'évolution}
\begin{block}{Améliorations futures}
\begin{itemize}
\item Application mobile (iOS/Android)
\item Tableau de bord analytique avancé
\item Notifications temps réel via WebSocket
\item Chatbot avec machine learning
\item Système de paiement en ligne
\item Système de notation et d'avis
\end{itemize}
\end{block}
\end{frame}

% ============================================
% FIN
% ============================================
\begin{frame}{}
\begin{center}
\Huge \textbf{Merci pour votre attention}

\vspace{1cm}

\Large Questions ?
\end{center}
\end{frame}

\end{document}

